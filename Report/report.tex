%%%%%%%% DATA LITERACY 2025 LATEX PROJECT TEMPLATE FILE %%%%%%%%%%%%%%%%%
%%% Based on the 2025 ICML template, available at https://icml.cc/Conferences/2025/AuthorInstructions %%%

\documentclass{article}

% Recommended, but optional, packages for figures and better typesetting:
\usepackage{microtype}
\usepackage{graphicx}
\usepackage{subfigure}
\usepackage{booktabs} % for professional tables

\usepackage{tikz}
% Corporate Design of the University of Tübingen
% Primary Colors
\definecolor{TUred}{RGB}{165,30,55}
\definecolor{TUgold}{RGB}{180,160,105}
\definecolor{TUdark}{RGB}{50,65,75}
\definecolor{TUgray}{RGB}{175,179,183}

% Secondary Colors
\definecolor{TUdarkblue}{RGB}{65,90,140}
\definecolor{TUblue}{RGB}{0,105,170}
\definecolor{TUlightblue}{RGB}{80,170,200}
\definecolor{TUlightgreen}{RGB}{130,185,160}
\definecolor{TUgreen}{RGB}{125,165,75}
\definecolor{TUdarkgreen}{RGB}{50,110,30}
\definecolor{TUocre}{RGB}{200,80,60}
\definecolor{TUviolet}{RGB}{175,110,150}
\definecolor{TUmauve}{RGB}{180,160,150}
\definecolor{TUbeige}{RGB}{215,180,105}
\definecolor{TUorange}{RGB}{210,150,0}
\definecolor{TUbrown}{RGB}{145,105,70}

% hyperref makes hyperlinks in the resulting PDF.
% If your build breaks (sometimes temporarily if a hyperlink spans a page)
% please comment out the following usepackage line and replace
% \usepackage{icml2023} with \usepackage[nohyperref]{icml2023} above.
\usepackage{hyperref}


% Attempt to make hyperref and algorithmic work together better:
\newcommand{\theHalgorithm}{\arabic{algorithm}}

\usepackage[accepted]{icml2025}

% For theorems and such
\usepackage{amsmath}
\usepackage{amssymb}
\usepackage{mathtools}
\usepackage{amsthm}

% if you use cleveref..
\usepackage[capitalize,noabbrev]{cleveref}

% Todonotes is useful during development; simply uncomment the next line
%    and comment out the line below the next line to turn off comments
%\usepackage[disable,textsize=tiny]{todonotes}
\usepackage[textsize=tiny]{todonotes}


% The \icmltitle you define below is probably too long as a header.
% Therefore, a short form for the running title is supplied here:
\icmltitlerunning{Project Report Template for Data Literacy 2025}

\begin{document}

\twocolumn[
\icmltitle{My Data Literacy Project\\ (Replace this with your Project Title)}

% It is OKAY to include author information, even for blind
% submissions: the style file will automatically remove it for you
% unless you've provided the [accepted] option to the icml2023
% package.

% List of affiliations: The first argument should be a (short)
% identifier you will use later to specify author affiliations
% Academic affiliations should list Department, University, City, Region, Country
% Industry affiliations should list Company, City, Region, Country

% You can specify symbols, otherwise they are numbered in order.
% Ideally, you should not use this facility. Affiliations will be numbered
% in order of appearance and this is the preferred way.
\icmlsetsymbol{equal}{*}

\begin{icmlauthorlist}
% Alphabetic order, equal contribution anyway.
\icmlauthor{Zhi Jing}{equal,first}
\icmlauthor{Enrico Quinto}{equal,second}
\icmlauthor{T\`ai Th\'ai}{equal,third}
\icmlauthor{Jonas Thumbs}{equal,fourth}
\icmlauthor{Baisu Zhou}{equal,fifth}
\end{icmlauthorlist}

% fill in your matrikelnummer, email address, degree, for each group member
\icmlaffiliation{first}{Matrikelnummer 12345678, MSc Machine Learning}
\icmlaffiliation{second}{Matrikelnummer 12345678, Exchange(?)}
\icmlaffiliation{third}{Matrikelnummer 12345678, MSc Machine Learning}
\icmlaffiliation{fourth}{Matrikelnummer 12345678, MSc Machine Learning}
\icmlaffiliation{fifth}{Matrikelnummer 7264384, MSc Machine Learning}

% put your email addresses here. You can use initials to save space, 
% e.g. if you are called Max Mustermann, you can use \icmlcorrespondingauthor{MM}{max.mustermann@uni-tuebingen.de}
% DO USE YOUR UNIVERSITY EMAIL ADDRESS!
\icmlcorrespondingauthor{ZJ}{first1.last1@uni-tuebingen.de} 
\icmlcorrespondingauthor{EQ}{first2.last2@uni-tuebingen.de}
\icmlcorrespondingauthor{TT}{first3.last3@uni-tuebingen.de}
\icmlcorrespondingauthor{JT}{first4.last4@uni-tuebingen.de}
\icmlcorrespondingauthor{BZ}{baisu.zhou@student.uni-tuebingen.de}

% You may provide any keywords that you
% find helpful for describing your paper; these are used to populate
% the "keywords" metadata in the PDF but will not be shown in the document
\icmlkeywords{Machine Learning, ICML}

\vskip 0.3in
]

% this must go after the closing bracket ] following \twocolumn[ ...

% This command actually creates the footnote in the first column
% listing the affiliations and the copyright notice.
% The command takes one argument, which is text to display at the start of the footnote.
% The \icmlEqualContribution command is standard text for equal contribution.
% Remove it (just {}) if you do not need this facility.

%\printAffiliationsAndNotice{}  % leave blank if no need to mention equal contribution
\printAffiliationsAndNotice{\icmlEqualContribution} % otherwise use the standard text.

\begin{abstract}
TODO.
\end{abstract}

\section{Introduction}\label{sec:intro}

Measuring color accurately is a common problem in daily life, e.g. when a surface needs to be repainted in the same color. Devices that can carry out this task - so-called \emph{colorimeters} - are available commercially and usually cost several hundred dollars. 

On the other hand, there are smartphone apps that claim to offer the same service for free by using the built-in camera. A major challenge in this approach is how to deal with the effect of ambient light. \textcolor{lightgray}{Simply picking the color from a photo is not ideal because hue, saturation and brightness depend heavily on the lighting conditions. One way to address this issue is to capture the color of a known object (e.g. a white piece of paper) at the same time and use this to correct for the ambient light.}

In this study, we want to find out the best approach to correct for ambient light and determine how accurate the measurements from such an app can be. In order to do this, we collected a dataset, explored several algorithms for color correction and calculated the accuracy of the measurements with respect to a suitable metric.

\section{Data and Methods}\label{sec:methods}

TODO.

\section{Results}\label{sec:results}

TODO.

\section{Discussion \& Conclusion}\label{sec:conclusion}

TODO.

\newpage

\section*{Contribution Statement}

TODO: Explain here, in one sentence per person, what each group member contributed.

% \section*{Notes} 

% Your entire report has a \textbf{hard page limit of 4 pages} excluding references and the contribution statement. (I.e. any pages beyond page 4 must only contain the contribution statement and references). Appendices are \emph{not} possible. But you can put additional material, like interactive visualizations or videos, on a githunb repo (use \href{https://github.com/pnkraemer/tueplots}{links} in your pdf to refer to them). Each report has to contain \textbf{at least three plots or visualizations}, and \textbf{cite at least two references}. More details about how to prepare the report, inclucing how to produce plots, cite correctly, and how to ideally structure your github repo, will be discussed in the lecture, where a rubric for the evaluation will also be provided.


\bibliography{bibliography}
\bibliographystyle{icml2025}

\end{document}

% This document was modified from the files available at https://icml.cc/Conferences/2025/AuthorInstructions
% the full copyright notice is available within the file icml2025.sty